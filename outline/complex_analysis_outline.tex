\documentclass[8pt,a4paper,fleqn]{article}
\usepackage{extsizes}
\usepackage[left=25mm,right=25mm,top=30mm,bottom=25mm]{geometry}
\usepackage[T1]{fontenc}
\usepackage[utf8]{inputenc}
\usepackage{amsmath,mathtools}
\usepackage{dsfont}
\usepackage{multicol}
\usepackage{times}

\newcommand\setComplex{\mathds{C}}
\newcommand\setReal{\mathds{R}}

\begin{document}
  \hrule
  \begin{center}
  \huge \textbf{Funktionentheorie: Gliederung}
  \end{center}
  \bigskip
  {\large \begin{minipage}[c]{0.4\textwidth}\flushleft Markus Pawellek \\ markuspawellek@gmail.com\end{minipage} \hfill \begin{minipage}[c]{0.4\textwidth}\flushright\today\end{minipage}}
  \medskip
  \hrule
  \bigskip

  \begin{multicols}{2}
  \section{Holomorphe Funktionen} % (fold)
  \label{sec:holomorphe_funktionen}

  \subsection{Komplexe Zahlen} % (fold)
  \label{sub:komplexe_zahlen}
  \begin{itemize}
    \item Definition, Eigenschaften und Funktionen
    \item Polarkoordinaten und andere Darstellungen
    \item Definition und Eigenschaften der Betragsfunktion
  \end{itemize}
  % subsection komplexe_zahlen (end)

  \subsection{Grenzwerte} % (fold)
  \label{sub:grenzwerte}
  \begin{itemize}
    \item Definition Grenzwert einer Folge von komplexen Zahlen
    \item Charakterisierung der Konvergenz
    \item Beispiel: Standardgrenzwert
  \end{itemize}
  % subsection grenzwerte (end)

  \subsection{Reihen} % (fold)
  \label{sub:reihen}
  \begin{itemize}
    \item Definition konvergenter Reihen
    \item Definition absolut konvergenter Reihen
    \item Beispiel: Potenzreihe
  \end{itemize}
  % subsection reihen (end)

  \subsection{Komplexe Funktionen} % (fold)
  \label{sub:komplexe_funktionen}
  \begin{itemize}
    \item Definition und Charakterisierung
    \item Beispiel
  \end{itemize}
  % subsection komplexe_funktionen (end)

  \subsection{Stetigkeit komplexer Funktionen} % (fold)
  \label{sub:stetigkeit_komplexer_funktionen}
  \begin{itemize}
    \item Definition und Charakterisierung
    \item Erhalt der Stetigkeit unter Bildung von Produkt, Quotient oder Komposition
    \item Bedingung für Stetigkeit des Betrages
    \item Beispiele
    \item Fortsetzung der Exponentialfunktion
    \item Eigenschaften der Fortsetzung
  \end{itemize}
  % subsection stetigkeit_komplexer_funktionen (end)

  \subsection{Potenzreihen} % (fold)
  \label{sub:potenzreihen}
  \begin{itemize}
    \item Definition
    \item Definition Häufungspunkt einer Folge
    \item Definition $\limsup$ und $\liminf$
    \item Beispiel: Häufungspunkte der angeordneten rationalen Zahlen
    \item Satz über Konvergenzradius einer Potenzreihe mit Grenzfällen
    \item Beispiele für die Berechnung des Konvergenzradius
  \end{itemize}
  \subsubsection{Eigenschaften} % (fold)
  \label{ssub:eigenschaften}
  \begin{itemize}
    \item Gleichmäßige Konvergenz
    \item Stetigkeit
  \end{itemize}
  % subsubsection eigenschaften (end)
  % subsection potenzreihen (end)

  \subsection{Differenzierbarkeit} % (fold)
  \label{sub:differenzierbarkeit}
  \begin{itemize}
    \item Definition
    \item Bemerkung Bedeutung und Betragsfunktion
    \item Satz über Erhaltung der Rechenregeln aus dem reellen Zahlenbereich
    \item Differenzierbarkeit impliziert Stetigkeit
    \item Definition holomorphe Funktion
    \item Beispiele holomorpher Funktionen
    \item Satz über die Holomorphie und Ableitung einer Potenzreihe
    \item Folgerung: Exponentialfunktion ist holomorph und ihre eigene Ableitung
    \item Folgerung: Potenzreihen sind beliebig oft differenzierbar
    \item Satz über die $n$.~Ableitung einer Potenzreihe
    \item Bemerkungen zur Umkehrung von Sätzen
  \end{itemize}
  % subsection differenzierbarkeit (end)

  \subsection{Die Cauchy-Riemann'schen Differentialgleichungen} % (fold)
  \label{sub:die_cauchy_riemann_schen_differentialgleichungen}
  \begin{itemize}
    \item Problemstellung
    \item Charakterisierung eindimensionale reelle Differenzierbarkeit in einem Punkt
    \item Definition zweidimensionale Differenzierbarkeit
    \item Satz über die Existenz der partiellen Ableitungen
    \item Satz über die Existenz der Ableitung bei stetigen partiellen Ableitungen
    \item Satz: partielle Ableitungen differenzierbarer Funktionen gehorchen den Cauchy-Riemann'schen Differentialgleichungen
    \item Satz: Komponenten holomorpher Funktionen sind harmonisch
  \end{itemize}
  % subsection die_cauchy_riemann_schen_differentialgleichungen (end)
  % section holomorphe_funktionen (end)

  \section{Der Cauchy'sche Integralsatz} % (fold)
  \label{sec:der_cauchy_sche_integralsatz}
  \subsection{Komplexe Kurvenintegrale} % (fold)
  \label{sub:komplexe_kurvenintegrale}
  \begin{itemize}
    \item Wiederholung Kurvenintegrale für das Riemann-Integral, Rektifizierbarkeit und Kurvenintegrale 2.~Art
    \item Idee der komplexen Kurvenintegrale
    \item Definition
    \item Einfache Eigenschaften
    \item Lemma: Separation, Orientierungsabhängigkeit und Beschränktheit
    \item Beispiel: Hyperbel um $z_0 \in \setComplex$
  \end{itemize}
  % subsection komplexe_kurvenintegrale (end)

  \subsection{Stammfunktionen} % (fold)
  \label{sub:stammfunktionen}
  \begin{itemize}
    \item Definition
    \item Beispiele
    \item Lemma: HDI für komplexe Stammfunktionen und Kurvenintegrale
    \item Exkurs über Kurvenintegrale 2.~Art
    \item Wegabhängigkeit des Kurvenintegrals 2.~Art
    \item Definition Gradientenfeld
    \item Lemma: Notwendige Bedingung für Gradientenfeld
    \item Definition sternförmiger Teilmengen des $\setReal^2$
    \item Beispiele sternförmiger Mengen
    \item Satz: Charakterisierung der Existenz eines Gradientenfelds
    \item Satz über Kurvenintegrale bei existierendem Gradientenfeld
    \item Beispiele
    \item Lemma über die Berechnung der Stammfunktion
  \end{itemize}
  % subsection stammfunktionen (end)

  \subsection{Der Cauchy'sche Integralsatz} % (fold)
  \label{sub:der_cauchy_sche_integralsatz}
  \begin{itemize}
    \item Lemma: Integrallemma von Goursat
    \item Satz: Integralsatz von Cauchy
    \item Bemerkung: Ersetzung der Sternförmigkeit durch einfach zusammenhängend
    \item Beispiel: Fresnel'sche Integrale
  \end{itemize}
  % subsection der_cauchy_sche_integralsatz (end)

  \subsection{Die Cauchy'sche Integrationsformel} % (fold)
  \label{sub:die_cauchy_sche_integrationsformel}
  \begin{itemize}
    \item Lemma: Gleichheit von geschlossenen Kurvenintegralen
    \item Satz: Identität des Kurvenintegrals
    \item Bemerkung
  \end{itemize}
  % subsection die_cauchy_sche_integrationsformel (end)
  % section der_cauchy_sche_integralsatz (end)

  \section{Eigenschaften holomorpher Funktionen} % (fold)
  \label{sec:eigenschaften_holomorpher_funktionen}
  \subsection{Potenzreihen} % (fold)
  \label{sub:potenzreihen}
  \begin{itemize}
    \item Satz: Holomorphie impliziert Potenzreihenentwicklung
    \item Folgerung: Holomorphie impliziert unendliche Differenzierbarkeit
  \end{itemize}
  % subsection potenzreihen (end)

  \subsection{Der Satz von Liouville} % (fold)
  \label{sub:der_satz_von_liouville}
  \begin{itemize}
    \item Definition ganzer Funktionen
    \item Bemerkungen zu Folgerungen
    \item Beispiele
    \item Satz: Beschränkung impliziert Polynom
    \item Folgerung: Fundamentalsatz der Algebra
  \end{itemize}
  % subsection der_satz_von_liouville (end)

  \subsection{Der Identitätssatz} % (fold)
  \label{sub:der_identitätssatz}
  \begin{itemize}
    \item Idee und Problem
    \item Definition zusammenhängender Teilmengen
    \item Satz: Charakterisierung der Gleichheit zwei holomorpher Funktionen
    \item Folgerung: Eindeutigkeit und Existenz holomorpher Funktionen
    \item Bemerkung über eindeutig bestimmte Fortsetzungen
  \end{itemize}
  % subsection der_identitätssatz (end)

  \subsection{Das Maximumsprinzip} % (fold)
  \label{sub:das_maximumsprinzip}
  \begin{itemize}
    \item Satz: Maximum impliziert Konstanz
    \item Folgerung: Betrag realisiert Maximum auf dem Rand
    \item Folgerung: Betrag realisiert Minimum auf dem Rand
  \end{itemize}
  % subsection das_maximumsprinzip (end)
  % section eigenschaften_holomorpher_funktionen (end)

  \section{Singularitäten- und Residuentheorie} % (fold)
  \label{sec:singularitäten_und_residuentheorie}
  \subsection{Laurentreihen} % (fold)
  \label{sub:laurentreihen}
  \begin{itemize}
    \item Satz: Berechnung der Potenzreihe holomorpher Funktionen
    \item Bemerkung: Laurentreihe, regulärer Teil und Hauptteil
    \item Satz: Koeffizienten der Potenzreihe sind eindeutig
  \end{itemize}
  % subsection laurentreihen (end)

  \subsection{Singularitäten} % (fold)
  \label{sub:singularitäten}
  \begin{itemize}
    \item Definition hebbare Singularitäten, Polstellen und wesentliche Singularitäten
    \item Beispiele
    \item Satz: Charakterisierung von Singularitäten
    \item Bemerkung
  \end{itemize}
  % subsection singularitäten (end)
  % section singularitäten_und_residuentheorie (end)
  \end{multicols}
\end{document}